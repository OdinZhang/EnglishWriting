\chapter{简单句}

\section{主谓}

句型是最基本的简单句句型,
由主语和谓语构成,
可简称为主谓结构。
主语通常是名词或代词,
谓语是不及物动词或动词短语。
这一句型还常带有状语,
用来说明时间、地点、目的、原因、方式或程度等。
句子的状语可以是副词、介词短语、
不定式(短语)、分词(短语)、
形容词短语、从句等。
例如:

\begin{enumerate}
	\item An orange moon rose behind the pine trees. \\ 橘黄色的月亮从松树后面升起。
	\item A glow of light appears over the sea. \\ 海上出现一束亮光。
\end{enumerate}

There + be结构是一种较常见的特殊句子结构,
可以看作是主谓句型的一种变体。
例如:
\vspace{1em}

There is not a day he spent with her that does not arouse sweet memory.

他同她一起度过的每一天都会唤起甜蜜的回忆。
\vspace{1em}

在There + be句型中,
谓语除用be外,
还可用appear, arise, come, go,
fall, keep, stay, enter, exist,
happen, lie, live, occur, remain,
rise, seem, stand等。例如:

\begin{enumerate}
	\item There appear to be several different ways to settle the problem. \\ 似乎有若干个方法解决这个问题。
	\item There existed different opioins on this question. \\ 对这个问题曾有过不同的看法。
\end{enumerate}

\subsection{改错}

\begin{enumerate}
	\item 旅游景点竭力满足游客的需求和品位。\\
	      Tourist spots devote to fulfilling tourists’ needs and tastes.\\
	      \emph{Tourists spots endeavor to cater for tourists’ needs and tastes.}
	\item 审美观因文化而异。\\
	      Perception of beauty is differed from culture to culture.\\
	      \emph{Perception of beauty differs from culture to culture.}
	\item 在大城市有一些贫困的社区。\\
	      In large –scale cities have some deprived communities.\\
	      \emph{There are some economically deprived communities in large cities.}
\end{enumerate}

\section{主系表}

在主系表句型中,系动词、主语的补足语或称表语,
构成“主—系—表”结构。
其特征是可用形容词作表语。
及物动词和不及物动词一般都不能有形容词紧随其后,
及物动词要有宾语,
不及物动词则经常有副词作状语,
这是区别于“主—系—表”结构的重要特征。
掌握这一点,
写作中就不会形容词、
副词乱用了。
除系动词be以外,常见的系动词如下:

\begin{enumerate}
	\item 表示“变得”、“成为”的系动词,如:\\ become, come, fall, get, go, grow, run, turn 等。
	\item 表示“保持着某一状态”的系动词,如:\\ continue, hold, keep, lie, remain, stand, stay 等。
	\item 表示“看起来”、“好像”的系动词,如:\\ appear, look, seem 等。
	\item 表示“感觉”的系动词,如:\\ feel, smell, taste, sound 等。
\end{enumerate}

表语可以是名词、形容词、分词、不定式、介词短语、从句等。例如:


\begin{enumerate}
	\item Our country is \emph{getting} more and more \emph{prosperous}. \\ 我们的国家日益繁荣。
	\item When the crops fail, the people \emph{go hungry}. \\ 收成不好,人们就挨饿。
\end{enumerate}

\subsection{改错}

\begin{enumerate}
	\item 贫穷是社区犯罪增多的原因。\\
	      The reason which cause increasingly number of community crimes in society is poverty.\\
	      \emph{Poverty is responsible for the crime wave in many communities.}
	\item 贫穷国家的首要问题是满足人们的基本生存需求。\\
	      Deprived countries concern how to satisfy citizens’ requirements of survive.\\
	      \emph{The top priority of deprived countries is to satisfy citizens’ basic needs.}
\end{enumerate}

\section{主谓宾}

在主谓宾句型中,
名词、代词、数词、动名词(短语)、
不定式(短语)或从句都可以在句子中充当宾语。
在宾语后常常可以带有修饰语。

\begin{enumerate}
	\item 在许多及物动词之后,
	      可接动名词作宾语。
	      能跟这种宾语的常见动词有avoid, admit,
	      appreciate, approve, consider,
	      deny, decline, enjoy, finish,
	      keep, mind, postpone, propose,
	      reject, require, resume,
	      risk, quit, recommend等。
	      例如:
	      \vspace{1em}

	      He is considering going abroad to further his education. \\ 他正在考虑出国继续深造。
	      \vspace{1em}
	\item 有些及物动词后接不定式作宾语,
	      如:afford, agree, ask, decide,
	      determine, hope, order, pretend,
	      try, promise, want, wish, learn,
	      desire, choose, expect等。
	      例如:\vspace{1em}

	      He has decided to give up the chance. \\ 他决定放弃这次机会。
	      \vspace{1em}
	\item 有些及物动词既可以接动名词也可以接不定式作宾语,
	      其意义基本相同,如begin, continue,
	      intend, like, prefer, start等。
	\item 有些动词如forget, need,
	      remember, stop, try,
	      mean等后接动名词或不定式作宾语时含义不同。
	      试比较:
	      \vspace{1em}

	      We must try to solve this problem.

	      We can try solving this problem in other ways.
	      \vspace{1em}
\end{enumerate}

\subsection{改错}

\begin{enumerate}
	\item 平等接受教育能帮助解决学生学习成绩不好的问题。\\
	      Equal access to education can overcome educational underachievement.\\
	      \emph{Equal access to education can help tackle educational underachievement.}
	\item 经常做运动会提高人的自信。\\
	      Exercise regularly can rise one’s confidence.\\
	      \emph{Regular exercise can increase one’s self-confidence.}
	\item 接触不同的文化可以促进创新。\\
	      Contacting with a wide variety of cultures can promote the creativity of native culture.\\
	      \emph{Exposure to different  cultures can encourage creativity.}
\end{enumerate}

\section{主谓双宾}

在主谓双宾句型中,
及物动词后接双宾语意义才能完整。
间接宾语指人,常用名词或代词表示。
直接宾语指物,通常用名词表示。
我们把这种句型称作主谓双宾结构。
常用于此句型的动词有:allow, bring,
buy, fetch, get, give, lend,
offer, save, sell, send, show,
take, wish等。
间接宾语有时也可以置于直接宾语之后,
间接宾语前需加介词to或for,
此时的to或for无实际意义,
只起连接作用。
例如:


\begin{enumerate}
	\item The winter in Harbin gives visitors an indelible impression of excitement.
	\item Everyone could offer suggestions to me.
\end{enumerate}

\subsection{改错}

政府应该给一些城市提供资金去保护历史建筑。\\
Many areas now pay attention to protect historic buildings.\\
\emph{Governments should offer some cities funds to preserve historic buildings.}

\section{主谓宾宾补}

在主谓宾宾补句型中,动词除需要接宾语外,
还应有宾语补足语,
才能使句子的意义表达完整。
宾语与其补足语一起构成复合宾语,
并在逻辑上构成主谓关系或系表关系。
我们把这种句型称作主谓宾补结构。

适于此句型的常见动词有:
believe, call, consider,
cut, elect, find, have,
keep, leave, let, like,
make, name, polish, think, want, wish等。

宾语补足语可以由名词、形容词、分词(短语)、
不定式(短语)、介词(短语)等构成。
用不定式作宾语补足语,表达和强调即将发生的事实;
用现在分词作宾语补足语,表示和强调正在进行的行为;
用过去分词作宾语补足语,表达和强调已发生的事实。
例如:


\begin{enumerate}
	\item We consider him a reliable friend. \\ 我们把他看作值得信赖的朋友。
	\item People often find economic systems extremely complicated. \\ 人们常觉得经济体系太复杂了。
	\item The police discovered the check hidden under a pile of papers. \\ 警察发现那张支票藏在一堆文件下面。
\end{enumerate}

\subsection{改错}


\begin{enumerate}
	\item 不健康的生活方式让人处于生病的危险之中。\\
	      Unhealthy lifestyle is likely to make people at risk of illness.\\
	      \emph{An unhealthy lifestyle is likely to put people at risk of illness.}
	\item 法律应该将醉驾定为刑事犯罪。\\
	      The government should introduce a law which is drunk driving a criminal offence.\\
	      \emph{Legislation should make drunk driving a criminal offence.}
\end{enumerate}

\section{总结}

\begin{tabular}[!htbp]{cc}
	\toprule
	主语或宾语   & 名词、代词、动名词、不定式(短语)、数词     \\ \midrule
	谓语         & 动词                                         \\ \midrule
	表语或补足语 & 名词、不定式(短语)、形容词、分词、介词短语 \\ \midrule
	定语         & 名词、代词、不定式(短语)、                 \\
	             & 形容词、分词、数词、介词短语                 \\ \midrule
	状语         & 不定式(短语)、分词、介词短语、副词         \\ \midrule
	同位语       & 名词、代词、动名词、不定式(短语)           \\ \bottomrule
\end{tabular}