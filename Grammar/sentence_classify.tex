\chapter{句子类别}

\section{简单句}

只包含一个主语和谓语的句子称为简单句。
但有时一个句子可包含两个或两个以上并列的主语或谓语,
还仍为简单句。
例如:

\begin{enumerate}
	\item Four years’ college life laid a solid foundation for his work. \\ 四年大学生活为他的工作打下了坚实的基础。
	\item The boys and girls jumped and played in the garden. \\ 男孩们和女孩们在花园里跳跃着、玩耍着。
\end{enumerate}

\section{并列句}

包含两个或两个以上互不依从的主谓结构的句子称为并列句。
在意义上,
各分句是同等重要并相互关联的;
在语法结构上,
它们是平行且相互间没有从属关系的。
英语并列句不能只用逗号隔开(较短的句子例外),
而要用分号或并列连词连接,
连词前可用或不用逗号。
另外,
一个句子中如果有两个以上的并列分句,
而且要用同样色并列连词时,
通常只在最后一个句子前用这个连词,
其他分句之间用逗号。
常见的并列连词有:and, but, yet, for, so,
while, whereas, or, as well as,
rather than, either\ldots or,
neither\ldots nor, not only \ldots but also\ldots 等。

\begin{enumerate}
	\item and 表示平行、顺接、转折、让步、对照、评注等\\
	      Her brother is an engineer \emph{and} her sister is a painter. (平行)\\
	      Alice is clever \emph{and} Jane is dull. (=but对比)\\
	      He didn’t come to the party, \emph{and} that’s a pity. (评注)\\
	      One more word \emph{and} I’ll knock you flat.(条件和结果)\\
	      He can’t keep the flowers alive \emph{and} he has watered them well, too.(= although he has watered\ldots)(让步)
	\item but和yet表示转折或对照(但是,然而)\\
	      It’s truth that he is young, \emph{but} he is experience and responsible.\\
	      He is poor, \emph{yet} he is clever and noble-hearted.\\
	      but和yet尽管都可以译为“但是”,区别还是有的。
	      \begin{enumerate}
		      \item but是并列连词,而yet则可作并列连词或副词,不可说and but,但可说and yet,but不可放在句尾,而yet则可放在句尾
		      \item but表示对照或对立时,一般都比较轻松自然,而yet表示对照或对立时,则往往比较强烈,时常出人意料。例如:\\
		            She is an American \emph{but} she speaks Chinese very fluently.  (自然轻松的比较或对立)\\
		            She is an American, \emph{yet} she knows little about American history. (强烈的比较或矛盾)
	      \end{enumerate}
	\item for表示原因或理由(因为)\\
	      It must have rained, \emph{for} the ground is wet.
	\item so表示结果(所以)\\
	      It is foggy today, \emph{so} we can’t see the distant hills.
	\item while, whereas表示对比(而)\\
	      Wise men seek after truth \emph{while} (whereas) fools despise knowledge.
	\item or表示选择(或者,不然的话)\\
	      Wear your coat \emph{or} you will catch cold.
	\item 有些并列连词也可以连接并列分句,如: either\ldots or, neither\ldots nor, not only\ldots but also\\
	      \emph{Either} we take part-time jobs to get to know the society, \emph{or} we can volunteer to participate in some social services.\\
	      \emph{Neither} I would consult him, \emph{nor} he would ask me for advice.\\
	      \emph{Not only} was the room well decorated, \emph{but also} meals was ready.
    \item besides等副词也可以起到并列连词的作用\\
    有些副词,
    如besides, consequently, furthermore
    等不是修饰句中的副词、形容词或动词,
    而是起承接作用,使上下文(句)意思连贯,
    语义衔接,形成逻辑性、连贯性良好的语篇。
    这副类词实际上是作连词用的,
    通常称为等立连接副词,可分为如下几类:
    \begin{enumerate}
        \item 表示意义增补、补充和说明的:further, furthermore, then, again, similarly, besides, additionally, moreover, likewise等
        \item 表示意思相反、对比的:contrarily, inversely, conversely, oppositely, rather等
        \item 表示内容与上文类似或相同的:similarly, identically, equivalently, equally, correspondingly, likewise等
        \item 表示概括或总结的:generally, overall, altogether
        \item 表示结果的:hence, thus, therefore, then, accordingly, consequently等
        \item 表示时间的:sometimes, meanwhile, occasionally
        \item 表示让步的:still, yet, nevertheless, however, though, notwithstanding等
        \item 表示条件的:else, otherwise 等
        \item 表示列举的:first (ly), second (ly), third (ly), finally等
    \end{enumerate}
\end{enumerate}

值得注意的是,
这些词所连接的并列分句一般要用分号隔开
(不用逗号,但可用句号),
但前面若有and, but,or等并列连词时,
则不用分号,要用逗号。如:

\begin{itemize}
    \item The rain was heavy; consequently the land was flooded.
    \item Girls wear fashionable clothes. Similarly, some birds have bright feathers.
    \item The weather changed suddenly, and accordingly we had to change our plan.
\end{itemize}

注:then指逻辑顺序,强调结果;consequently强调因果关系;therefore和hence表示严格的推理,后面的结论是必然结果;accordingly表示自然的结果。

\section{复合句}

由一个主句和至少一个从句构成的句子称为复合句。
在复合句中,
主句是整个句子的主体部分,
从句只是整个句子的一个成分,
不能独立存在,
在句法关系上从属于主句。
从句须用关联词引导,
来表明与主句的关系。
一般可分为三类:名词性从句,定语从句和状语从句。

\subsection{名词性从句}

%ppt不完整,等下次上课